%! Author = breandanconsidine
%! Date = 8/22/21

% Preamble
\documentclass[12pt]{article}

% Packages
\usepackage{amsmath}
\usepackage{listings}
\usepackage{fontspec}
\usepackage{xcolor}

\setmonofont{JetBrains Mono}[
  Contextuals=Alternate  % Activate the calt feature
]

\makeatletter
\def\verbatim@nolig@list{}
\makeatother

\lstdefinelanguage{kotlin}{
  comment=[l]{//},
  commentstyle={\color{gray}\ttfamily},
  emph={delegate, filter, firstOrNull, forEach, it, lazy, mapNotNull, println, repeat, assert, with, head, tail, len, return@},
  numberstyle=\noncopyable,
  emphstyle={\color{OrangeRed}},
  identifierstyle=\color{black},
  keywords={abstract, actual, as, as?, break, by, class, companion, continue, data, do, dynamic, else, enum, expect, false, final, for, fun, get, if, import, in, infix, interface, internal, is, null, object, open, operator, override, package, private, public, return, sealed, set, super, suspend, this, throw, true, try, catch, typealias, val, var, vararg, when, where, while, tailrec, reified},
  keywordstyle={\color{NavyBlue}\bfseries},
  morecomment=[s]{/*}{*/},
  morestring=[b]",
  morestring=[s]{"""*}{*"""},
  ndkeywords={@Deprecated, @JvmField, @JvmName, @JvmOverloads, @JvmStatic, @JvmSynthetic, Array, Byte, Double, Float, Boolean, Int, Integer, Iterable, Long, Runnable, Short, String},
  ndkeywordstyle={\color{BurntOrange}\bfseries},
  sensitive=true,
  stringstyle={\color{ForestGreen}\ttfamily},
  literate={`}{{\char0}}1
}

\title{A Simple Yet Effective Method Slicer}
\author{Breandan Considine}
% Document
\begin{document}
  \maketitle
We describe a simple heuristic for extracting method slices in well-formed source code using a Dyck counter.~\footnote{\url{https://en.wikipedia.org/wiki/Dyck\_language}} A common coding convention is to prefix functions with a keyword, followed by a group of balanced brackets and one or more blank lines. We observe this pattern can be used to slice methods in a variety of programming langauges. An implementation in Kotlin is given below, which will output the following source code when run on itself:

  \vspace{11pt}

  \begin{lstlisting}[basicstyle=\footnotesize\ttfamily, language=kotlin,label={lst:lstlisting}]
fun String.sliceIntoMethods(kwds: Set<String> = setOf("fun ")) =
  lines().fold(-1 to List<String>(0)) { (dyckCtr, methods), ln ->
    if (dyckCtr < 0 && kwds.any { it in ln }) {
      ln.countBalancedBrackets() to (methods + ln)
    } else if (dyckCtr == 0) {
      if (ln.isBlank()) -1 to methods else 0 to methods.put(ln)
    } else if (dyckCtr > 0) {
      dyckCtr + ln.countBalancedBrackets() to methods.put(ln)
    } else -1 to methods
  }.second

fun List<String>.put(s: String) = dropLast(1) + (last() + "\n$s")

fun String.countBalancedBrackets() = fold(0) { sum, char ->
  val (lbs, rbs) = setOf('(', '{', '[') to setOf(')', '}', ']')
  if (char in lbs) sum + 1 else if (char in rbs) sum - 1 else sum
}

fun main(args: Array<String>) =
  println(args[0].sliceIntoMethods().joinToString("\n\n"))
  \end{lstlisting}

\end{document}